\documentclass[sigconf]{acmart}

\usepackage{hyperref}

\usepackage{endfloat}
\renewcommand{\efloatseparator}{\mbox{}} % no new page between figures

\usepackage{booktabs} % For formal tables

\settopmatter{printacmref=false} % Removes citation information below abstract
\renewcommand\footnotetextcopyrightpermission[1]{} % removes footnote with conference information in first column
\pagestyle{plain} % removes running headers

\begin{document}
\title{Big Data Platforms as a Service}


\author{Tiffany Fabianac}
\orcid{}
\affiliation{%
  \institution{Indiana University}
  \streetaddress{}
  \city{Bloomington} 
  \state{Indiana} 
  \postcode{}
}
\email{tifabi@iu.edu}

% The default list of authors is too long for headers}
\renewcommand{\shortauthors}{T. Fabianac}


\begin{abstract}
This paper uses an industry example of a large pharmaceutical client to explore the problems faced to implementing big data platform solutions and the benefits these solutions offer once in use.

\end{abstract}

\keywords{Big Data, Platform, Cloud Architecture}


\maketitle

\section{Introduction}
Most pharmaceutical companies have adopted one or many Laboratory Information Management Systems (LIMS) and/or Electronic Laboratory Notebooks (ELN). These systems are often implemented as standalone systems within a single Research and Development (R\&D) group or even within a single laboratory. A problem seen in large- or mid-sized pharmaceutical companies is that different research groups within the same organization often implement different LIMS or ELN. This severely restricts data sharing and reuse between groups which leads to many problems including the same experiment being run multiple times between different groups, regulatory inefficiencies in tracking sample use and storage, and bottle necked development cycles due to missing data. 

One of the emerging strategies to combat the problems arising from isolated systems is to combine systems using cloud computing. Platform as a Service (PaaS) provides an environment for the development and execution of applications and software tools. The platform is the heart of a cloud computing infrastructure that enables software on-top as well as data created from such software to be accessed and used my a multitude of users\cite{Ojala}.

This review seeks to outline the benefits and challenges of using a PaaS approach to share and regulate R\&D data within a large pharmaceutical company that has already implemented numerous laboratory systems. 

\section{Importance of Platforms}
%why is this topic important
Many organizations struggle with the aim of sharing data and processing tools among researchers. SaaP provides a method of better resource utilization while reducing maintenance costs\cite{Oh}. 

\section{Implementing Platforms}
%cover issues
The overarching concern with storing data outside of the organization is security. Numerous methods have been developed to assure cloud security such as integrated stacks used by Google and Microsoft Azure and Service Level Agreements (SLAs)\cite{Casola}.
%explain stacks and SLA and STAR


\section{Platforms and Big Data}
%how it's relevant to big data


\begin{acks}

  The authors would like to thank Dr. Gregor Von Laszewski  and Teaching Assistants Saber Sheybani and Miao Jiang.

\end{acks}

\bibliographystyle{ACM-Reference-Format}
\bibliography{report} 

\end{document}

